% !Mode:: ``TeX:UTF-8''
\documentclass[a4paper,9pt,titlepage]{article}

\usepackage[T1]{fontenc}
\usepackage{lmodern}
\usepackage[spanish,es-lcroman]{babel}
\usepackage{ucs}
\usepackage{amsmath}
\usepackage{multicol}
\usepackage{alltt}
\usepackage{verbatim}
\usepackage{graphicx}
\usepackage{eso-pic}  % For background PDF.
\usepackage{amssymb}
\usepackage{amsthm}
\usepackage{ifthen}
\usepackage{xcolor}
\usepackage{minted}
\usepackage{xspace}   % Para remover espacios al final de una palabra.
\usepackage{array}
\usepackage{url}
\usepackage{cleveref}
\usepackage{subfig}
\usepackage{pdfpages}
\usepackage{eso-pic}
\usepackage[absolute]{textpos}
\usepackage{extsizes} % Permite utilizar más variedad de tamaños de tipografía.
\usepackage{setspace} % Para el interlineado.
\usepackage{epstopdf} % Para incorporar el logo en EPS.
\usepackage{pdfpages} % Incluir PDFs en el documento (para la portada).
\usepackage{mdwlist}  % Para usar itemize*, enumerate*, etc.
\usepackage[bottom]{footmisc}
\usepackage{enumitem}
\usepackage{geometry}
\usepackage{fancyhdr} % Para varias cuestiones de estilos.
\usepackage{titlesec}
\usepackage{verbatimbox}
\usepackage{comment}
\usepackage{upquote}
\usepackage{fontspec}
\usepackage{changepage}  % To change margins in the middle of a page.
\usepackage{tabto}
\usepackage{ifthen}

\definecolor{bletras}{RGB}{54,125,162}

\geometry{
    paper=a4paper,
    left=6cm,
    right=2cm,
    top=2cm,
    bottom=2.5cm,
}

\AddToShipoutPictureBG{
  \put(0,0){\includegraphics{inc/background.pdf}}%
}

\setmainfont{Source Sans Pro}
\spacing{1.1}                     % Interlineado por defecto.
\setlength{\parindent}{0em}   % Quitar la sangría al inicio de cada nuevo párrafo.
\setlength{\parskip}{0.6em}

% Support for fancy styles
\pagestyle{fancy}
\fancyhf{}
\renewcommand{\headrulewidth}{0pt}

% Pie de página.
\fancyfoot[R]{
  \fontsize{7}{11}\selectfont
  \vspace{3mm}
  \NombreCompleto\ \Apellidos. Año \the\year. Pág. \thepage. \\
}

\setlist[enumerate,2]{leftmargin=1.2cm,label=\alph*.,ref=\alph*}
\setlist[enumerate,3]{label=\roman*.,ref=\roman*}
\setlist[itemize,1]{leftmargin=1.2cm}

% Variables principales.
\def  \NombreCompleto   {Leandro Damián}
\def  \Apellidos        {Di Tommaso}
\def  \Direccion        {Ruta 215 Km. 28,500}
\def  \Localidad        {La Plata (1901) - Buenos Aires}
\def  \Correo           {leandro.ditommaso@mikroways.net}
\def  \SitioWeb         {http://www.mikroways.net}

% Encabezado de la práctica.
\newcommand{\Encabezado}{
                        \begin{adjustwidth}{-40mm}{}
                        \begin{flushleft}
                          \vspace*{-7mm}
                          \hspace{3mm}\LARGE\textbf{\color{white}\NombreCompleto}
                          \LARGE\textbf{\color{bletras}\Apellidos}\\
                        \end{flushleft}
                        \end{adjustwidth}
                        \begin{adjustwidth}{-60mm}{137mm}
                          \begin{flushright}
                          \vspace{6mm}
                          \small\color{white}
                          \textbf{\Direccion}\\
                          \vspace{-0.5mm}
                          \textbf{\Localidad}\\
                          \vspace{-0.5mm}
                          \textbf{\Correo}\\
                          \vspace{-0.5mm}
                          \textbf{\SitioWeb}\\
                          \vspace{-16mm}
                          \end{flushright}
                        \end{adjustwidth}
                        }

% Títulos.
\newcommand{\Seccion}[1]{
  \vspace{1em}\uppercase{\section*{{#1}}}{\titlerule[0.8pt]}\vspace{0.6em}
}

\titleformat{\section}{\fontsize{10}{10}\bfseries}{\thesection}{0em}{}

\titlespacing{\section}{0em}{0em}{0em}

\newcommand{\Posicion}[3]{
  \vspace{-0.8em}
  \subsection*{
    \color{white}\tabto{-4.5cm}\uppercase{\textit{#1}}\color{bletras}\tabto{0cm}
    \uppercase{#2}\ifthenelse{\equal{#3}{}}{.}{; {#3}.}
  }
  \vspace{-0.2em}
}

\titleformat{\subsection}
  {\fontsize{9}{10}\bfseries}
  {
    \begin{adjustwidth}{-60mm}{}
      \begin{flushleft}
        \thesubsection
      \end{flushleft}
    \end{adjustwidth}
  }
{1em}{}

\titlespacing{\subsection}{0em}{0em}{0em}


\newcommand{\Idioma}[1]{
  \vspace{0.5em}\uppercase{\subsubsection*{#1}}\vspace{-0.5em}
}

\titlespacing{\subsubsection}{0em}{0em}{0em}

\input{inc/environments.tex}


\begin{document}
\begin{cv}

\Seccion{Perfil}

Responsable, autodidacta y proactivo, amante de los desafíos y la generación de soluciones
creativas e innovadoras. Con facilidad para relacionarme en grupo, así como para exponer
ideas, proyectos y puntos de vista frente a una audiencia. Tengo especial interés en
mantener un constante camino de aprendizaje y superación personal y profesional.

Actualmente, me encuentro trabajando en soluciones de DevOps, con herramientas como Chef,
Docker y Rancher, y en tareas de monitoreo de servicios y aplicaciones a través de la
generación de estadísticas y el análisis de los datos obtenidos. Mi principal área de
experiencia está dada en el diseño, implementación, administración y resolución de
problemas en servicios sobre equipos de tipo Unix.

Manejo fluido del idioma inglés que me permite mantener conversaciones, tanto orales como
escritas.

\Seccion{Experiencia laboral}

\Posicion{2016.03 - Presente}{MIKROWAYS}{La Plata}

Socio Gerente en empresa dedicada a brindar soluciones de TI en áreas como DevOps,
monitoreo, IoT y consultoría y asesoramiento en soluciones tecnológicas.

\Posicion{2014.05 - Presente}{INTA}{Ciudad Autónoma de Buenos Aires}

Consultor, desarrollando tareas relacionadas al área de infraestructura y redes,
trabajando actualmente en el rediseño e implementación de los ambientes productivos, de
pre-producción y de desarrollo de servicios web basados en Linux con características de
alta disponibilidad y escalabilidad, utilizando un esquema de balanceo de los diferentes
servicios involucrados. Implementando para los mismos una infraestructura de orquestación
de servidores basada en Chef.

\Posicion{2013.01 - Presente}{SENASA}{Ciudad Autónoma de Buenos Aires}

Consultor, desarrollando diversas tareas orientadas principalmente al área de servidores.
Definición de proyectos para el organismo y colaboración en la definición y la
implementación de proyectos existentes. Creación de una infraestructura para la
utilización de Chef, definición de recetas generales para los servidores y específicas
para el servicio de correo, habiendo realizado una reingeniería del mismo y la
reinstalación utilizando Chef. Implementación de servicio de versionado de código con GIT.
Documentación de diferentes servicios y actualización de documentación existente.

\Posicion{2006.03 - Presente}{UNIÓN OBRERA METALÚRGICA}{La Plata, Buenos Aires}

Administrador de red, servidores y equipos del organismo.

\Posicion{2014.08 - 2016.12}{CeSPI, UNLP}{La Plata, Buenos Aires}

DevOp, responsable de la administración de la infraestructura de sistemas en producción de
la UNLP.

\Posicion{2014.02 - 2015.06}{LA BIOGUÍA}{Comunidad web}

Diseño, dimensionamiento y optimización de servidor web utilizando una arquitectura basada
en Varnish, Nginx, PHP-FPM y MySQL sobre una plataforma Linux para permitir el acceso
concurrente de los clientes necesarios. Finalizado el trabajo inicial, se continúa
brindando soporte y asistencia en el crecimiento e implementación de nuevas tecnologías
necesarias para acompañar el servicio.

\Posicion{2013.06 - 2014.08}{FECLIBA}{La Plata, Buenos Aires}

Consultor y soporte, desarrollando tareas relacionadas al área de infraestructura y redes.
Algunas de las tareas llevadas a cabo son la implementación de una infraestructura de
virtualización con Proxmox, virtualización de varios servidores, implementación de
servicio de backups, reinstalación de servicio de correo y DNS.

\Posicion{2013.04 - 2013.10}{MINISTERIO DE INDUSTRIA DE LA NACIÓN}{Ciudad Autónoma de
Buenos Aires}

Implementación de infraestructura de virtualización basada en Proxmox, virtualización de
varios servidores importantes del organismo e instalación de algunos servicios
adicionales.

\Posicion{2013.02 - 2013.05}{INTA}{Ciudad Autónoma de Buenos Aires}

Implementación de Microsoft Active Directory Federation Services para proveer el soporte
de Single Sign On para las aplicaciones del organismo.

\Posicion{2012.04 - 2012.12}{VECTUS SRL}{Ciudad Autónoma de Buenos Aires}

Empleado por Vectus SRL, trabajando on-site en INTA Central como parte del equipo de
Infraestructura y Redes, que tiene a cargo la administración de la LAN de Buenos Aires y
la WAN de todo el organismo, así como los servicios informáticos que brinda el mismo.

\Posicion{2007.07 - 2012.04}{CESPI, UNLP}{La Plata, Buenos Aires}

Miembro del equipo de trabajo encargado de la administración del NOC de la red y
servidores en producción de la Universidad Nacional de La Plata.

\Posicion{2010.06}{SUMED}{La Plata, Buenos Aires}

Instalación y configuración de servidor Web/VPN/LDAP basado en Linux.

\Posicion{2010.05 - 2010.09}{SERVICIO METEOROLÓGICO NACIONAL}{Ciudad Autónoma de Buenos
Aires}

Reingeniería de servidores del organismo, virtualización, redundancia y reinstalación de
los siguientes servicios: base de datos (MySQL), bridge transparente para administración
de ancho de banda (OpenBSD con PF), correo electrónico, DNS, FTP, LDAP, Proxy y otros
sistemas internos. 

\Posicion{2009.11}{COMPLEJO TURÍSTICO DUNAS Y MAR}{Mar Azul, Buenos Aires}

Instalación y configuración de equipos inalámbricos para conectar diferentes edificios y
cubrir con wifi las cabañas del complejo.

\Posicion{2009.06 - 2009.09}{LEGISLATURA PORTEÑA}{Ciudad Autónoma de Buenos Aires}

Auditoría de seguridad de la infraestructura de red para el proyecto de voto electrónico.

\Posicion{2008}{FACULTAD DE INFORMÁTICA, UNLP}{La Plata, Buenos Aires}

Diseño y configuración de Hotspot para el nuevo edificio utilizando equipos Mikrotik.

\Posicion{2008}{UNIVERSIDAD NACIONAL GENERAL SARMIENTO}{Los Polvorines, Malvinas
Argentinas, Buenos Aires}

Reinstalación de servidor Proxy/Firewall/Gateway.

\Posicion{2008}{HIPERMERCADO MAYORISTA NINI}{La Plata, Buenos Aires}

Reinstalación de servidor Proxy/Firewall/Gateway.

\Posicion{2008}{DECANATO DE CIENCIAS EXACTAS, UNLP}{La Plata, Buenos Aires}

Instalación de servidor de correo electrónico basado en Linux y con autenticación
utilizando LDAP.

\Posicion{2008}{FACULTAD DE ARQUITECTURA Y URBANISMO, UNLP}{La Plata, Buenos Aires}

Reinstalación y migración de servidor de archivos utilizando Samba, perteneciente a la
oficina de alumnos.

\Seccion{Experiencia docente}

\Posicion{2017.03 - PRESENTE}{FACULTAD DE INFORMÁTICA, UNLP}{La Plata, Buenos Aires}

Colaborador de la materia Seminario de Lenguajes C.

\Posicion{2015.10 - PRESENTE}{FACULTAD DE INFORMÁTICA, UNLP}{La Plata, Buenos Aires}

Jefe de Trabajos Prácticos de la materia Redes y Comunicaciones.

\Posicion{2016.11}{VI ENCUENTRO NACIONAL DE TÉCNICOS, CABASE}{Villa Gesell, Buenos Aires}

Exposición titulada “De Desarrollo a Producción usando Docker”.

\Posicion{2014.06 - 2015.09}{FACULTAD DE INFORMÁTICA, UNLP}{La Plata, Buenos Aires}

Ayudante diplomado de la materia Redes y Comunicaciones.

\Posicion{2011.05 - 2014.05}{FACULTAD DE INFORMÁTICA, UNLP}{La Plata, Buenos Aires}

Ayudante alumno de la materia Redes y Comunicaciones.

\Posicion{2013.08 - 2013-12}{FACULTAD DE INFORMÁTICA, UNLP}{La Plata, Buenos Aires}

Dictado especial de la materia Seminario de Redes para la Promoción de Egreso del plan 90.

\Posicion{2010.04 - 2011.05}{FACULTAD DE INFORMÁTICA, UNLP}{La Plata, Buenos Aires}

Colaborador de la materia Redes y Comunicaciones.

\Posicion{2010.04}{BOLDT SA}{La Plata, Buenos Aires}

Curso de Linux, nivel inicial, preparado especialmente para adaptarlo a las necesidades de
la empresa.

\Posicion{2009}{FACULTAD DE INFORMÁTICA, UNLP}{La Plata, Buenos Aires}

Exposición titulada “Software Libre para la Administración de Redes”, como parte de las
Jornadas de Software Libre preparadas por la Facultad.

\Posicion{2008.08 - 2010.12}{ACADEMIA CESPI, UNLP}{La Plata, Buenos Aires}

Instructor de los cursos Cisco CCNA y CCNA Security.

\Seccion{Formación académica}

\Posicion{2017}{Licenciatura en Sistemas, plan 2007; Facultad de Informática, Universidad
Nacional de La Plata}{}

\Posicion{2013}{ANALISTA PROGRAMADOR UNIVERSITARIO, PLAN 2007; FACULTAD DE INFORMÁTICA,
UNIVERSIDAD NACIONAL DE LA PLATA}{}

\Seccion{Idiomas}

\Idioma{Inglés}

\Posicion{2005.12}{FIRST CERTIFICATE IN ENGLISH, CAMBRIDGE UNIVERSITY; INSTITUTO
BRITÁNICO}{La Plata, Buenos Aires}

\Posicion{2006 - 2012}{CLASES PARTICULARES DE GRAMÁTICA, ESCRITURA, LECTURA Y
CONVERSACIÓN}{}

\Posicion{2000 - 2004}{CURSO REGULAR DE JÓVENES; INSTITUTO BRITÁNICO}{La Plata, Buenos
Aires}

\Idioma{Alemán}

\Posicion{2011.08 - 2012.04}{CLASES PARTICULARES DE GRAMÁTICA Y CONVERSACIÓN; INSTITUTO
CULTURAL ARGENTINO-ALEMÁN}{La Plata, Buenos Aires}

\Idioma{Italiano}

\Posicion{2008.08 - 2008.12}{CLASES PARTICULARES DE GRAMÁTICA Y CONVERSACIÓN}{}

\Idioma{Portugués}

\Posicion{2007.08 - 2007.10}{CLASES PARTICULARES DE GRAMÁTICA Y CONVERSACIÓN}{}

\Seccion{Cursos, seminarios y talleres}

\Posicion{2017.09}{Introducción a la ciencia de datos; FACULTAD DE INFORMÁTICA,
UNIVERSIDAD NACIONAL DE LA PLATA}{}

Carga horaria: 40 horas. Otorga 3 créditos para el Doctorado en Ciencias Informáticas.

\Posicion{2017.06}{CURSO RUTEO AVANZADO; FACULTAD DE INFORMÁTICA, UNIVERSIDAD NACIONAL DE
LA PLATA}{}

Carga horaria: 70 horas. Otorga 4 créditos para el Doctorado en Ciencias Informáticas.

\Posicion{2017.02}{JORNADAS DE CAPACITACIÓN DOCENTE; IBM ARGENTINA}{}

Carga horaria: 15 horas. Temas abordados: Big Data \& Analytics, Design Thinking.

\Posicion{2016.11}{INTRODUCCIÓN A LOS PROCESOS DE ENSEÑAR Y APRENDER; FACULTAD DE
INFORMÁTICA, UNIVERSIDAD NACIONAL DE LA PLATA}{}

Carga horaria: 25 horas.

\Posicion{2011.10}{REDES WIRELESS; FACULTAD DE INFORMÁTICA, UNIVERSIDAD NACIONAL DE LA
PLATA}{}

Carga horaria: 20 horas. Dictado en el marco del Congreso Argentino de Ciencias de la
Computación (CACIC).

\Posicion{2010.10}{CURSO DE IPV6 BÁSICO; NIC.BR}{San Pablo, Brasil}

Carga horaria: 42 horas.

\Posicion{2010.08}{CISCO CCNP ROUTE VERSIÓN 6; FUNDACIÓN PROYDESA}{Ciudad Autónoma de
Buenos Aires}

Carga horaria: 45 horas.

\Posicion{2010.08}{CISCO CCNP ROUTE VERSIÓN 6; FUNDACIÓN PROYDESA}{Ciudad Autónoma de
Buenos Aires}

Carga horaria: 45 horas. Capacitación como instructor.

\Posicion{2010.02}{CISCO CCNA SECURITY; FUNDACIÓN PROYDESA}{Ciudad Autónoma de Buenos
Aires}

Carga horaria: 45 horas. Capacitación como instructor.

\Posicion{2010.02}{CISCO CCNA VERSIÓN 4.0, ACCESO A LA WAN; FUNDACIÓN PROYDESA}{Ciudad
Autónoma de Buenos Aires}

Carga horaria: 45 horas. Capacitación como instructor.

\Posicion{2009.09}{TALLER DE PROGRAMACIÓN DE DRIVERS EN LINUX; FACULTAD DE INFORMÁTICA,
UNIVERSIDAD NACIONAL DE LA PLATA}{}

Carga horaria: 6 horas. Docente: Christoph Hellwig.

\Posicion{2009.07}{CISCO CCNA VERSIÓN 4.0, CONMUTACIÓN Y CONEXIÓN INALÁMBRICA DE LAN;
FUNDACIÓN PROYDESA}{Ciudad Autónoma de Buenos Aires}

Carga horaria: 45 horas. Capacitación como instructor.

\Posicion{2009.05}{CURSO OFICIAL DE MIKROTIK, NIVEL AVANZADO}{Ciudad Autónoma de Buenos
Aires}

Carga horaria: 32 horas. Docente: Maximiliano Dobladez, Alessio Garavano.

\Posicion{2009.03-2009.07}{CISCO CCNA VERSIÓN 3.1, TECNOLOGÍAS WAN; ACADEMIA CESPI,
UNIVERSIDAD NACIONAL DE LA PLATA}{}

Carga horaria: 72 horas.

\Posicion{2009.02}{CISCO CCNA VERSIÓN 4.0, ROUTERS Y PRINCIPIOS BÁSICOS DE ENRUTAMIENTO; 
FUNDACIÓN PROYDESA}{Ciudad Autónoma de Buenos Aires}

Carga horaria: 45 horas. Capacitación como instructor.

\Posicion{2008.08-2008.12}{CISCO CCNA VERSIÓN 3.1, PRINCIPIOS BÁSICOS DE CONMUTACIÓN Y
ENRUTAMIENTO INTERMEDIO; ACADEMIA CESPI, UNIVERSIDAD NACIONAL DE LA PLATA}{}

Carga horaria: 72 horas.

\Posicion{2008.08}{CISCO CCNA VERSIÓN 4.0, ASPECTOS BÁSICOS DE NETWORKING; FUNDACIÓN
PROYDESA}{Ciudad Autónoma de Buenos Aires}

Carga horaria: 45 horas. Capacitación como instructor.

\Posicion{2008.03-2008.07}{CISCO CCNA VERSIÓN 3.1, PRINCIPIOS BÁSICOS DE ROUTERS Y
ENRUTAMIENTO; ACADEMIA CESPI, UNIVERSIDAD NACIONAL DE LA PLATA}{}

Carga horaria: 72 horas.

\Posicion{2007.08-2007.12}{CISCO CCNA VERSIÓN 3.1, CONCEPTOS BÁSICOS SOBRE NETWORKING;
ACADEMIA CESPI, UNIVERSIDAD NACIONAL DE LA PLATA}{}

Carga horaria: 72 horas.

\Posicion{2006.02}{CURSO DE FIREWALLS; LINUX A DISTANCIA (LAD)}{}

Carga horaria: 12 horas cátedra.

\Posicion{2004.04-2004.06}{CURSO DE REDES INFORMÁTICAS; FUNDACIÓN UNIÓN DE CENTROS
EDUCATIVOS (FUCE), PROYECTO DE CAPACITACIÓN PLATENSE (PROCAP)}{}

Carga horaria: 120 horas cátedra.

\Posicion{2003.07-2003.11}{CURSO DE ARMADO Y REPARACIÓN DE PC; FUNDACIÓN UNIÓN DE CENTROS
EDUCATIVOS (FUCE), PROYECTO DE CAPACITACIÓN PLATENSE (PROCAP)}{}

Carga horaria: 120 horas cátedra.

\Posicion{2003.06}{CURSO DE MÉTODOS DE ESTUDIO; RESIDENCIA UNIVERSITARIA AULA NUEVA}{La
Plata, Buenos Aires}

Carga horaria: 12 horas.

\Seccion{Otras actividades}

\Posicion{}{BLOG SOBRE REDES, SEGURIDAD E INFRAESTRUCTURA}{}

URL: http://www.mikroways.net/blog



\end{cv}
\end{document}

